% ****** Start of file apssamp.tex ******
%
%   This file is part of the APS files in the REVTeX 4.1 distribution.
%   Version 4.1r of REVTeX, August 2010
%
\documentclass[reprint,%article
 %,
%superscriptaddress,
%groupedaddress,
%unsortedaddress,
%runinaddress,
%frontmatterverbose, 
%preprint,
%showpacs,preprintnumbers,
%nofootinbib,
%nobibnotes,
%bibnotes,
 amsmath,amssymb,
 aps,
%pra,
%prb,
%rmp,
%prstab,
%prstper,
%floatfix,
]{revtex4-1}

\usepackage{xcolor}

\usepackage{graphicx}% Include figure files
\usepackage{dcolumn}% Align table columns on decimal point
\usepackage{bm}% bold math
\usepackage{appendix}
\usepackage{subcaption}
%\usepackage{subfloat}
%\usepackage{hyperref}% add hypertext capabilities
%\usepackage[mathlines]{lineno}% Enable numbering of text and display math
%\linenumbers\relax % Commence numbering lines

%\usepackage[showframe,%Uncomment any one of the following lines to test 
%%scale=0.7, marginratio={1:1, 2:3}, ignoreall,% default settings
%%text={7in,10in},centering,
%%margin=1.5in,
%%total={6.5in,8.75in}, top=1.2in, left=0.9in, includefoot,
%%height=10in,a5paper,hmargin={3cm,0.8in},
%]{geometry}https://www.overleaf.com/project/5ba0361831a91c7b3990c0d7
 \usepackage{natbib}
 
\begin{document}

\preprint{APS/123-QED}

\title{DRAFT- Rotation Curve Fitting Model }% Force line breaks with \\
%\thanks{ Thanks Jesus}%

% Find out if Joe and Noah still want to be on this paper%

 

\author{Sophia Natalia Cisneros}
 \affiliation{ Department of Astronomy,  
 University of Washington}
 % \email{sophia.cisneros@du.edu}
  \author{Richard Ott}%
 \email{rich.ott@EMAIL.com}
\affiliation{Gotta get that}%Tech dudes \textbackslash\textbackslash
%}
 \author{ Meagan Crowley}
\email{ }
\affiliation{NREL}%
 \author{ Amy Roberts}
\email{ }
\affiliation{CU Denver}%
\author{Marcus Paz}
\author{Zaneeyiah Brown}
\author{Landon Joyal}
\author{Roberto Real Rico}
\author{Elizabeth Gutierrez-Gutierrez}
\author{ Phong Pham}
\author{Zac Holland}
\author{Amanda Livingston}
\author{Lily Castrellon}
\author{Shanon Rubin}
\author{Aaron Ashley}
\author{Dillon Battaglia}
%\affiliation{UMass Boston% with \\
%&}%
%
 
 
\date{\today}% It is always \today, today,
             %  but any date may be explicitly specified
             %%%%%%
%%%%%%%
%%%%%%
%%%%%%%
%%%%%%
%%%%%%%
\begin{abstract}
 
 
{\color{teal}  EDIT    }

\begin{description}
\item[Usage]
Secondary publications and information retrieval purposes.
\item[PACS numbers]
May be entered using the \verb+\pacs{#1}+ command.
%\item[Structure]
%You may use the \texttt{description} environment to structure your abstract;
%use the optional argument of the \verb+\item+ command to give the category of each item. 
\end{description}
\end{abstract}

\pacs{Valid PACS appear here}% PACS, the Physics and Astronomy
                             % Classification Scheme.
%\keywords{Suggested keywords}%Use showkeys class option if keyword
                              %display desired
\maketitle

%\tableofcontents
%%%%%%NOTES TO DO
%%%make labels for figure axes. remove orange line through the data
%%%%%%
%%%%%%%%
%%%%%%
%%%%%%%
%%%%%%
%%%%%%
\section{Introduction  \label{sec:uno}}


%DARK MATTER

 The flat-rotation curve problem of spiral galaxies  is primary evidence for dark matter\cite{Rub,Bosma,1985ApJAlbada}. 
 In the dark matter paradigm, this problem is   viewed as a missing mass component due to a conflict between kinematic phrasing of  two      observations of light, spectra and photometry. Photometry yields Keplerian orbital velocities, which fall-off outside of the stellar light. Doppler shifted spectra yields rotation velocities from the Lorentz Doppler shift formula  (see Fig.~\ref{fig:NGC2403}). The Doppler shifted spectra velocities, do not fall-off in radius but rather remain roughly constant to the limit of our observations. 
 
 
 
Dark matter theories are built from classical gravity, but with massive
exotic,     ''dark'' particles  which do not interact electromagnetically.   These particles have   yet to be    discovered, but are hypothesized to have a very low interaction probability with baryonic matter, and so  experiments to detect them are ever increasing their sensitivity \cite{Cebrian:2022brv}. However, in the absence of a definitive detection, the paradoxes of dark matter models are interesting. The first paradox is best stated by 
  S. McGaugh  when they ask   ``Why is the luminous tail wagging the dark matter dog,  if dark matter dominates dynamics ?'' \cite{1999McGaugh,McGaugh2016RAR}. This is known as the  disk-halo conspiracy,
that knowledge of 
 the luminous stellar   disk seems to completely determine the spherical dark matter halo \cite{2004ApJ...609..652M}. 
 
 
 
 
 A second paradox  is known as  the Universal Rotation Curve(URC)   (see Fig.~\ref{fig:URC}) \cite{salucci,Persic,1978Rubin,10.1111/j.1365-2966.2007.11696.x}, which  shows that small, gas dominated dwarf galaxies require   proportionally larger dark matter halos than   large,  bright  galaxies.  Classically mass accretion rates are proportional to total mass (CITE), not inversely proportional  \cite{10.1093/mnras/stt2403}.  To match this phenomenology, dark matter models must  fine-tune extra free-parameters,  which  reduces their current  predictivity   \cite{MCGAUGH2021220}. In the URC, 
 a spectrum of 1,100 galaxies    appears to inflect  about  the Milky Way's assumed   peak  rotation curve velocity. (SEE FIGURE)
 We will here interpret this as an indication of a frame-dependency in the problem due to our Milky Way. 
 
      
 
  
  %MOND   
    
  The leading   alternative theory  to dark matter,  Modified Newtonian Dynamics (MOND) \cite{Milgrom}. MOND proposes    
  that there is  a changing  physical law of gravitational acceleration   on the vast length scales of galaxies, and that the   luminous mass is the only mass
  \cite{McGaugh_2014}. 
  This approach  is   predictive across a large sample of  rotation curves, and   across a broad   range  of galaxy sizes and morphologies \cite{2016Lelli}. MOND uses only   a single free-parameter in fits to  galaxy rotation curves, which notably  has   a roughly constant value across all all spiral   galaxies.  \cite{McGaugh2016RAR,2022A&A...664A..40M}. 
  Previous explorations of  extending MOND into the relativistic regime have given rise to a new theories of gravity  \cite{PhysRevD.70.083509,doi:10.1142/S0217751X0703666X}, but we instead transition the concept of  changing acceleration scales to   the relativistic idea of  changing relative curvatures.   
  We propose  the     MONDian success at fitting galaxies  is due to a very effective characterization
  of the Milky Way's gravitational potential as a function of radius  from the data. 
   
 
 
 
 
%RCFM  


     Previously, the roll of 
   relative galaxy curvatures  in this problem  were    obviated 
       by  Galilean subtraction of   gravitational red-shifts at the  large $r$  limit of the data \citep{MTW}. 
       We   
      instead   borrow heavily from Einstein and Weyl's field frames concept, to parametrize   Lorentz-type frame transformations     between galaxies, one-to-one in radii,   with gravitational potentials. 
      In this paper we demonstrate the success of this approach by developing a fitting formalism and fitting   a sample of 174 galaxy rotation curves previously fitted by the MOND and dark matter models  \cite{McGaugh2016RAR,2016Lelli}, so that reduced $\chi^2$ values can be directly compared.  
      
   At this time, there are many problems in   the concordance model of   cosmology which are attributed to dark matter \cite{2010dmp..book.....S}.
   In
 this paper we address only  one,   the flat-rotation curve problem of rotationally supported galaxies. This is because  the  frame dependent symmetry   is particularly clear, and   naturally extends Modified Newtonian Dynamics concepts into the  relativistic framing with no need for  modification of classical gravity.
      We propose   that there are other places in our cosmology where dark matter is invoked which   may require different physics assumptions, as informed by observations of large scale flows \cite{Tully:2014gfa},    ``too'' early   formation of galaxies \cite{Naidu_2022}, discrepancies in estimates of the Hubble Flow from supernovae versus from the cosmic microwave background (CITE) and (WHAT"S THE OTHER ISSUE YO). 
      We do not present a fundamental derivation of our fitting formulae, but rather heuristically replace dark matter in the standard rotation curve formula with relative galaxy   curvatures due to luminous mass.  
      MOND  also   has a fascinating, compact explanation for another phenomenological symmetry of the flat-rotation curve problem, the  
   Tully-Fisher relation \cite{1977A&A....54..661T,McGaugh_2000}  which   emphasizes the predictive power of their approach.  We will attempt to extend the  framing presented here to the Tully-Fisher relation in a subsequent publication. 
 %In example, the James Webb Space Telescope may have already falsified dark matter driven galaxy formation \cite{2022arXiv221014915H}.
   
   
      
  
 
 
 
This paper  is organized as follows;
{\bf Section \ref{sec:dos}} constructs  the rotation curve  fitting formula, 
{\bf Section \ref{sec:data}} we describe  the data we use, 
 {\bf Section \ref{sec:analysis}} we  discuss our results, and explore a correlation of our free parameter to a ratio of galaxy density gradients, 
 and  {\bf Section \ref{sec:conclu}} we discuss implications for future tests.   
  
  
 \begin{figure}[h!]
     \centering
     \includegraphics[width=\linewidth]{URC}
     \caption{\emph{Universal Rotation Curve spectrum, Used with permission from Ref.\citep{salucci}}}
     \label{fig:URC}
\end{figure}
  

  
    
 \begin{figure}[h!]
%\scalebox{0.25075}%
      \centering
      \includegraphics[width=\linewidth]{NGC2403_deBlokThings_XueSofue}
      \caption{\emph{Rotation Curve of NGC 2403 \cite{Blok1}.   Rotation curve data blue dots with  error bars,  Keplerian velocity from luminous mass estimated by   green line,   RCFM fit blue line.} }
      \label{fig:NGC2403}
  \end{figure}
%%%%%%
%%%%%%%%
%%%%%%
%%%%%%%
%%%%%%
%%%%%%
\section{Rotation Curve Fitting Formulae  \label{sec:dos}}
 \subsection{Dark matter rotation curve fitting formula}
 
  

 The   dark matter rotation curve formula   is of the form

 \begin{equation}
v(r)^2_{obs}  =  v(r)^2_{lum}  +  v(r)^2_{dm},   
\label{eq:zonte1}
\end{equation} 

  where all velocities are assumed to be circular orbits about the rotation axis of the galaxy at  $r=0$, in the plane of the galactic disk. 
Terms in  $v_{obs}$ are velocity parameters   from Minkowski spacetime Lorentz boosts 
   

 \begin{equation}
 \frac{v_{obs} }{c}=
\frac{  \left( \frac{\omega'(r)}{\omega_o}\right)^2 -  \left( \frac{\omega_o}{\omega'(r)} \right)^2 }{  \left( \frac{\omega'(r)}{\omega_o}\right)^2  +  \left( \frac{\omega_o}{\omega'(r)}\right)^2 },
\label{eq:modelLumA}
\end{equation} 

 for    
 the rest-frame frequency $\omega_o$, the  observed  Doppler shifted frequency $\omega'$, and $c$  the constant vacuum light speed.  
 Terms in  $v_{lum}$ come from observations of total light  (photometry) interpreted by Population Synthesis Models (PSM) as mass,   hence orbital velocities  
  
   \begin{equation}
v(r)_{lum}^2 = \gamma_b v(r)_{bulge}^2 +  \gamma_d v(r)_{disk}^2 + v(r)_{gas}^2.    
\label{eq:zonte3}
\end{equation} 
  
 Terms in    $\gamma_i$  are the mass-to-light ratios for the stellar bulge $\gamma_b$ and disk $\gamma_d$ respectively, which come from PSM. Gas measurements do not require the introduction of a mass-to-light ratio due to different measurement techniques which   require information regarding the column depth of the gas (ASK STACY OR RENE).  
 Terms in $v^2_{dm}$ represent
the dark matter, and  are  the algebraic difference of the   terms  $v^2_{obs}-v^2_{lum}$. 

  Quadratic terms in  Eq.~\ref{eq:zonte1} and Eq.~\ref{eq:zonte3}  reflect a   sum of gradients in the potential as a function of  radius, 
 for a central gravitational    force law   

\begin{equation}
 \frac{\partial \Phi(r)_{lum}}{\partial r}    =\frac{v(r)_{lum}^2}{r},  
    \label{zoochance1}
\end{equation}

  
   
and a  Newtonian gravitational potential 

\begin{equation}
      \Phi(r)  = -G \int d^3r'  \frac{ \rho(r') }{r-r'} ,
      \label{eq:Newt}
      \end{equation}

which solves Poisson's equation

\begin{equation}
\nabla^2 \Phi(r)_{lum}  = 4\pi G \rho(r).   
    \label{whatsgood}
\end{equation}

 Where $G$ is Newton's   gravitation constant, and 
$\rho(r')$  the mass density. 
  



\subsection{New Rotation Curve Fitting Formula}

  

 

 To test the   
frame-dependence picture, 
we  replace the gradient in the potential attributed to   dark matter      with  Lorentz-type boosts between galaxy frames parametrized with small deviations from flatness.   We    assume  luminosity   is a Lorentz scalar, therefore invariant under the assumption of a good distance estimate, and hence a faithful tracer of baryonic mass. 
We assume   
   Doppler shifted spectra is part of a Lorentz 4-vector, and so    must transform in a Lorentzian sense.
In addition, we 
   assume   shifts in   spectra   due to relative velocity and relative acceleration are separable \cite{Jack,Cisn}, perhaps roughly  as in Eq.~\ref{eq:zonte1}. 
   In what follows, all   terms  can be assumed to  be functions of radius except the model's free parameter $\alpha$,  which is single valued for each galaxy fitted. 
   
   The new rotation curve formula we propose  is
   

\begin{equation}
v_{rc}^2 =  v_{lum}^2+\alpha \kappa^2 v_{1} v_{2},  
\label{eq:zonteLCM}
\end{equation}  

for   a free parameter $\alpha$,  
$\kappa$  a curvature ratio 

 \begin{equation}
\kappa=\frac{\Phi_{gal}}{\Phi_{mw}}, 
\label{eq:kappa2}  
\end{equation}  

 with $\Phi_{gal}$ the    Newtonian gravitational potential of the galaxy being observed, and $\Phi_{mw}$ that of  the Milky Way. 


 Terms in $v_1$   are   
 
   \begin{equation}
       v_1 = \sinh \zeta. 
   \end{equation}
 
 for a rapidity angle $\zeta$ defined by the    Lorentz exponential  term  
  
   
     \begin{equation}
     e^{\zeta}=  \frac{\omega_{mw}}{\omega_{gal}}  =\sqrt{\frac{g_{tt}|_{gal}}{g_{tt}|_{mw}}},
      \label{eq:gravRS}
    \end{equation}
    
which  is our field-frame relationship between two galaxies, for  clock terms $g_{tt}$   defined in the   weak field Schwarzschild limit  \cite{Hartle}, 
 
  \begin{equation}
      g_{tt}= -( 1 - 2\Phi/ c^2).
      \label{clocktime}
  \end{equation} 
  








The second Lorentz-type term is  

\begin{equation}
v_{2} =  \cosh \tau, 
\label{eq:hyperbolico}
\end{equation}

 which  transforms between the  curved 2-frame   in  Eq.~\ref{eq:gravRS} and  the flat 2-frame where we make observations. 
 That this second transformation  is necessary is evidenced by the local constancy of the speed of light. 
 
 The   $v_2$   Lorentz exponential term  is a  convolution of  the curved and flat frames
 
\begin{equation}
    e^{\tau}=   e^{(\zeta+\eta)/2},
\end{equation}
 
for the  flat field-frame
Lorentz exponential  

\begin{equation}
    e^{\eta}=\frac{\omega_{l}}{\omega_o}= \sqrt{\frac{1+\beta}{1-\beta}}.   
    \label{eq:flat}
\end{equation}  
     
Terms in 
$\beta = v_{lum}/c$ are the
Keplerian rotation velocities  estimated from total light  $v_{lum}$, and  associated with  frequency shifts $\omega'_{l}$      by a Lorentz boost   

 \begin{equation}
 \frac{v_{lum} }{c}=
\frac{  \left( \frac{\omega'_{l}}{\omega_o}\right)^2 -  \left( \frac{\omega_o}{\omega'_{l}} \right)^2 }{  \left( \frac{\omega'_{l}}{\omega_o}\right)^2  +  \left( \frac{\omega_o}{\omega'_{l}}\right)^2 }. 
\label{eq:lumlorentz}
\end{equation} 
 
 
We assume Keplerian rotation curves are     the best estimate of flatness, since dark matter is not required to  reproduce the rotation curve of our Solar System.  
 
 
 Note, when viewed as    Rindler's accelerated coordinates\cite{MTW,Wald, rindler2013essential}, $v_1$ is  timelike   and $v_2$ is spacelike. 
    
  
 
  Fitting details and analysis are reported in Sec.~\ref{sec:analysis}. This concept has been previously explored in a series of papers \cite{Cisneros:2013vha,Cisneros:2014fea,Cisneros2015,Cisn2016}, though we here report an improved class of Lorentz-type transformations and a larger, more representative sample of galaxies. 
   
 


 





  
  
  
  
   
    
  
 
%%%%%%
%%%%%%%
%%%%%%
%%%%%%%  
%%%%%%
%%%%%%%  
\subsection{Gravity Details \label{sec:gravDets}}


 

   We     assume   the additional symmetry of the  Tetrad-formalism \cite{BertschingerClassTetrads}, which attaches  a    local Lorentz frame   at each point in the manifold 


\begin{equation}
    g^{\mu \nu} = \eta^{\alpha \beta} e^\mu_\alpha  e^\nu_\beta
\end{equation} 

such that the orthonormal  basis vectors  $e^\mu_\alpha$ carry the information of the physical spacetime, and the tetrad at each point is Lorentzian and so can endure 3 boosts and 3 rotations \cite{BertschingerClassTetrads}.  
(ASK CB or NAIR - this sound wrong, or read WALD - the basis is the $e^\mu$ and these are $\sqrt{g_{tt}}$ so they not flat, but inner product of one of those bad boys with the inverse covector of the other galaxy, would give us eq. 10, also checkiyo if the indices are correct, originally had a, b lower indices on the basis. CHECK) 
 
   
  Lorentz exponential terms (Eq.~\ref{eq:gravRS} and Eq.~\ref{eq:flat})
  are    identified  as the field frame relationship by  comparison of 
     the  Lorentz transformation in two forms, Eq.~\ref{eq:modelLumA} and
 the    hyperbolic form\cite{rindler2013essential} 


     \begin{equation}
         \frac{v}{c} = \tanh \theta = \frac{e^\theta - e^{-\theta}}{e^\theta + e^{-\theta}} .   
         \label{boost}
     \end{equation} 

 
 
  
The standard Schwarzschild gravitational redshift formula

 \begin{equation}
       \frac{\omega_1}{\omega_2}  =\sqrt{\frac{g_{tt}|_{P2}}{g_{tt}|_{P1}}} =\sqrt{\frac{|\xi^t\xi_{t}|_{P2}}{|\xi^t\xi_{t}|_{P1}}}. 
      \label{eq:grav}
    \end{equation} 

is a relationship between two points on a single manifold, but  to extend this formalism to compare two distinct galaxy manifold  we look   to   Wolfgang Rindler's  statement that    ``the center of each galaxy provides a basic local standard of nonacceleration ... so then can be treated like a local inertial frame relative to its own center.''\cite{rindler2013essential}.

 
  Based on   Rindler's statement, 
  we relate two such Killing fields  by    synchronizing  the galaxy centers such that the  frames can then be compared inertially and admit  Lorentz-type transformations. 
  The derivation of Eq.~\ref{eq:grav} came from the conservation properties of  the timelike  Killing vector fields  
   $\xi^t(r)$, for which the norm of the field is the time  component of the metric $\xi^b \xi_b =g_{tt}$ \cite{Wald}. 
 Gravitational manifolds foliated in
 Terms in  $g_{tt}$ rely upon evaluation of the Newtonian  gravitational potentials about the galaxy center, so to
 synchronize   clock fields    is to synchronize 
   the Newtonian  gravitational potentials  $\Phi $. 
   
   
 Classically, $\Phi$ is integrated from the large $r$ limit of the data  $r \to \infty$,  into the   small $r\approx 0$ limit. 
This integration order is an implicit assumption of a flat embedding spacetime at $\infty$, so that at the large $r$ limit  all potentials go to zero and energy is conserved. We drop this constraint, since a priori the external environments of galaxies a the limit of the data do not appear to be    flat, 
and in fact are    extremely   diverse \cite{Pomarede:2020pme,Hoffman:2017ako}. What is more,  we have no definite idea of the curvature of the universal embedding spacetime, the background vacuum energy content of the universe (CITE), or the  global   value of the Hubble constant(CITE).  

%what we do know
What we do know, to high confidence,     is that all galaxy clocks read the same time of $t=0$ at the event horizon. On the scales of this problem observationally, the event horizon is essentially at $r=0$. 
%Since physically  what we measure    is differences in potentials, our ignorance is of no account if we synchronize galaxy centers. 
  In practice,  this   means we  reverse the 
    order of integration of $\Phi$,   from the small $r$ limit out  to the large $r$  extent of the data  

 \begin{equation}
     \Phi  =   \int^{R-big}_{r-small} \vec{F_r}\cdot\vec{dr}  . 
      \label{eq:Newt2}
      \end{equation}
 
   Since what we measure physically is differences in potential, this means we are comparing all galaxies from their centers, which is where  Rindler said  they can all  be treated as inertial frames in comparison to each other and  Lorentz-type transformations can be used to compare them in  $g_{tt}(r)$. 
  
  
  Potentials calculated in this way still obey the central force law for test particles moving in circular orbits in Eq.~\ref{zoochance1} and Poisson's equation Eq.~\ref{whatsgood}.
   
  

   
 
 

  
 

%%%%%%
%%%%%%%
%%%%%%
%%%%%%%

%%%%%%
%%%%%%
%%%%%%% 
%%%%%%
%%%%%%
%%%%%%%  
%%%%%%
%%%%%%%  
%%%%%%
%%%%%%%
\section{Data \label{sec:data}}

{\color{red}  }
{\color{red} \rule{\linewidth}{0.5mm}}
\subsection{SPARC dataset}
 
 We fit a sample  of  175 nearby galaxies with extended rotation curves from atomic hydrogen (HI) collected from the literature and reported and standardized in  the SPARC data base library (Spitzer Photometry and Accurate Rotation Curves\cite{2016Lelli}). HI provides the most reliable
 rotation curves because it dynamically cold, traces circular orbits, and can be observed several effective radii past the stellar disk. 
 This sample of rotationally supported galaxies also spans the widest range of masses and morphologies currently available. In addition, these galaxies are  accompanied by mass models which come from surface photometry in the 
   near infrared  at 3.5$\mu m$, which is widely believed to be the best tracer of stellar mass and in population synthesis models  gives mass-to-light (M/L) ratios which are almost constant independent of star formation history,  versus those from   optical measurements\cite{BelldYong,10.1093/mnras/sty3223}.  Stellar M/L ratios   translate between photometry and dynamics. 
   
   Mass models include Luminous mass contributions from a stellar bulge, stellar disk, gas fraction, as shown in Eq.~\ref{eq:zonte3}. Gas fractions are calculated from surface density profiles of HI by the formula from (CITE Casertano (1983)) which solves the Poisson equation for a disk geometry and
   includes the total mass of the HI scaled by a factor of 1.33 to include helium abundances . Stellar bulges are assumed to be spherical, and stellar disks are assumed to go to exponentially approach zero at  infinity unless there is a clear delineation of the end of the galaxy disk. It is interesting to note that \citet{2016Lelli} state that errors come from uncertainties in the M/L ratios rather than the geometry assumptions
   
 
%The gas contribution is calculated using the formula from Casertano (1983), which solves the Poisson equation for a disk
%with finite thickness and arbitrary density distribution. We
%assume a thin disk with a total mass of 1.33 M H I, where the
%factor 1.33 considers the contribution of helium. Vgas is either
%omputed using H I surface density profiles or taken from
%published mass models. For D512-2, D564-8, and D631-7,
%H I surface density profiles are not available; hence, we
%compute Vgas assuming an exponential profile with a scale
%ength of 2Rd. This is a reasonable approximation for such latetype galaxies (e.g., Swaters et al. 2002). We neglect the
%contribution of molecular gas because CO data are not
%available for most SPARC galaxies. 
   
 %Their final science sample is made of 153 galaxies.
 
 %Disk-Halo degeneracy: McGaugh says that van Albada talks about it here \cite{1985ApJAlbada}, but really just means the conspiracy between the peak rotation curve velocity and the luminous mass - and that it's hard/impossible to tell if the inner velocity is due to the stars or the halo. This is why Stacy uses the near infrared $3.6 \mu m$ wavelength that is the best tracer for stellar mass. Stellar Population Synthesis models suggest that in this wavelength band the M/L is essentially constant across a wide range of galaxy morphologies and  masses, independent of star formation history .
 

 
 van Albada: Assume a galaxy is composed of a stellar bulge and a disk, and assume that each has a constant M/L ratio. At 1/3 the effective radius (half light radius) the maxima of the rotation curve happens for spheroid and disk. 
   with  the RCFM in order to compare results with    MOND and RAR and dark matter fits to the same data set   \cite{McGaugh_2014,Li_2018}. 


%%%%%%
%%%%%%
%%%%%%%
\subsection{Milky Way}
 
All terms in the RCFM presented here require a static assumption of the Milky Way  baryon profile. 
Determination of the gravitational potential of the luminous mass in the Milky Way (MW) galaxy is a famously under-constrained problem of   taking observations of the system, from within the system. Many clever approaches have been used to constrain the    problem \cite{1991ARA&A..29..409F}.
However, outside of our position at 8 kpc in the disk of the MW, the observations have been notoriously difficult to interpret. In  recent large surveys of millions of  stars in the Milky Way  we begin to see  constraints to these models from  SDSS APOGEE-2 \cite{2022ApJS..259...35A},  \cite{10.1093/mnras/stt814}.  

However, in the absence of a definitive conclusion to the story of the mass of the MW, we choose three different static MW baryon distributions from \cite{Xue}, \cite{Sofue}, and CITE(MCGAUGH-figure out where he p ublished the MW stuff). 
Other static MW profiles are available at the Github repo: Cisneros-Galaxy/RCFM. 

\cite{2008ApJ...673..864J} Mario Juric's paper - has info  on the MW from SDSS, surveyed like 48,000 stars. Gotta get him to look at our MW models. 


 

IN \cite{Li2016ModellingMD} the bulge is more like Sofue. Not flat like McGaugh. 
To understand why. 
``using the recently released Gaia billion-star map8
, we propose a
Galactic disk mass distribution model which is based on the star density distribution
rather than the brightness and mass-to-light ratio. ''
TROUBLE: ``we obtain a flat rotation curve
which reproduces the key observed features with no need for a dark halo''.

  
\subsection{Geometry and Mass-to-light ratios}
Photometry 
 is a measure of   a  galaxy's luminosity, which   after processing   through a Population Synthesis Model \cite{BelldYong,10.1093/mnras/sty3223}, gives an estimate of the  baryonic mass components of the galaxy. These mass components, stellar bulge and disk and gas halo, then     yield   Keplerian velocities 
   from Poisson's equation, which fall-off outside of the stellar contributions. 
   
   
 All terms in $\Phi(r)$ used in this paper  are    integrated from estimates of the baryonic mass, as reported in the      SPARC  library \cite{2016Lelli}.  
   Assumptions in their reported luminous mass models  are included in Sec.~\ref{sec:analysis}
Studies of the mass-to-light ratios in galaxies began with Schwarzschild's early investigations \cite{1954AJ.....59..273S} and have evolved into a set of best practices, but modeling of the 
dark matter halo remains severely   under-constrained with correlated parameters including radius, slope, scale length, and halo core radius.     
The technique presented here, whether fundamental or no,  has the advantage of replacing the dark matter term  with a product of terms based only on estimates of the luminous mass and   one free parameter.


It is commonly assumed that   the stellar bulge, gas halo and   dark matter halo are spherically symmetric, but that the stellar disk is axially symmetric. We use 
the Schwarzschild metric   for clarity of the construction;  a static, stationary, spherically symmetric solution of Einstein's equations.
Flattened disks produce gravitational potentials which are weaker than a sphere of the same mass\cite{Chatterjee}. However, numerical integration of the disk is a computationally intensive and requires assumptions of  boundary conditions,   relevant physical scales,  etc. which add extra free-parameters\cite{2011A&A...531A..36H}. We 
  introduce the error of spherical symmetry   for the disk to maintain a clear presentation in the plane of the galactic disk, acknowledging  that inside of $r< R/3$, we   overestimate the gravitational potential, and  mass-to-light ratio by approximately $15\%$ (see Fig.~\ref{fig:my_geom}),  where R is the radius of the stellar disk configuration.  (CHECK WITH STACY). 


This introduced error does not change  the line shape, and   does not introduce any artifacts outside of $R/3$, where rotation curves become flat,  as the potentials for disk and sphere are commensurate there. The assumption of spherical symmetry is commonly used   in evaluation of the   rotation curve velocities from integrated potentials \cite{2022A&A...664A..40M,PhysRevD.70.083509}. 
 
\begin{figure}
    \centering
     \includegraphics[width=\linewidth]{Chatterjee_SphereDisk.png}
    \caption{ $\alpha = r/R$, for $R$ the radial extent of the stars \cite{Chatterjee}}
    \label{fig:my_geom}
\end{figure}

  
  \subsection{Thing}
We will parametrize gravitational frames from estimates of luminous mass reported kinematically as  $v_{lum}$, Eq.~\ref{eq:zonte1}. These velocities  are constructed from photometric observations of surface brightnesses, interpreted   by a Population Synthesis Model (PSM)\cite{10.1093/mnras/sty3223} as     masses. 
PSM rely upon a complex  suite of  assumptions regarding galaxy evolution, metallicities and initial mass functions  \cite{BelldYong,10.1093/mnras/sty3223}. PSM predict   mass-to-light ratios  $\gamma_i$
 which translate  luminosity into mass as in Eq.~\ref{eq:zonte3}. 
Gas fractions are  not scaled because the observation techniques do not require a PSM to yield masses.  Baryonic mass contributions (disk, bulge and gas) are summed quadratically to represent a mass sum in the same way as in   Eq.~\ref{eq:zonte1}.


  
 
%%%%%%
%%%%%%
%%%%%%%
\section{Analysis and Constraining the free parameter \label{sec:analysis}}

%%TO DO: PUT ASSUMPTIONS FOR SPARC LUM MODELS - " Assumptions in their reported luminous mass models  are included in Sec.~\ref{sec:analysis}. "

In RCFM fits reported here, the    bulge and disk mass-to-light ratios are allowed to vary freely (See Eq.~\ref{eq:zonte3}), though the average values are within stated criteria   \cite{2016Lelli} of $\pm 20\%$. The gas fractions (HI scaled for Helium abundance) are fixed though addition of molecular gas could increase mass fractions in the inner kiloparsec of a galaxy(CITE RENE WB), and it is common (CHECK-Y-YO) to scale the observed 21-cm flux by $4/3$ to account for molecular gas \cite{2004ApJ...609..652M}.


\subsection{NGC 3198 and other galaxies}
\cite{1985ApJAlbada} studies this galaxy, which MOND has troubles with. This galaxy has an exponential disk, they use 2 components -  a thin exponential disk de Vaucouleurs 1959;
Freeman 1970, and dark matter spheroid Young (1976). Dijo algo about using an infinite disk - uggg. 
 
\cite{Toky} mentions that MOND fails to fit NGC 3109, which we fit exceedingly well ($\chi^2 = 0.32$).
%%%%%%
%%%%%%%%
%%%%%%
%%%%%%%
%%%%%%
%%%%%%
\subsection{Correlation to  the model's free parameter }


 In \cite{2016Lelli}, it says that they have fifty  galaxies with highly accurate distance: (45) from tip of the red giant branch, (3) from Cepheids, (2) from supernova, with errors in distance on the order of $5\% - 10\%$. We use this subset to investigate   the free parameters for these galaxies with respect to a guess at its physical significance. 
 
 
 %NEED TO SEE IF WANT TO INCLUDE SN GALS TOO... train the model's free parameter we select a subset of galaxies who have Cepheid or Tip of the Red Giant Branch distance measurements. 
 % DO WE WANT TO DO THIS, OR JUST DO THAT SUBSEQUENT?> When the free parameter is fixed to a constant value, we will then run on the entire SPARC data set,
We also exclude galaxies  which have an inclination greater than $85^o$ as impossible to constrain a proper surface brightness profile, and those at inclinations less than $35^o$ as being impossible to accurately report line of sight Doppler shifts.   
  By this criteria, we select 46 galaxies (See Table \ref{tab:Tset}. 
  
  
  \begin{table*}[]
      \centering
      \begin{tabular}{|c|c|c|c|c|c|}
      \hline \hline
Galaxy Name & Hubble Type(1)& 	Distance (Mpc)&Mean Error on D (Mpc)& 	Distance Method (2)& 	Inc (deg)\\
    \hline \hline\\
CamB&   	10&    3.36&  	0.26&   2&  65\\
D564-8& 	10& 	8.79& 	0.28& 	2& 	63\\
D631-7& 	10& 	7.72& 	0.18& 	2& 	59\\
DDO154& 	10& 	4.04& 	0.2& 	2& 	64\\
DDO168& 	10& 	4.25& 	0.21& 	2& 	63\\
ESO444-G084& 10& 	4.83& 	0.48& 	2& 	32\\
IC2574& 	9& 	3.91& 	0.2&    	2& 	75\\
NGC0024& 	5& 	7.3& 	0.36&   	2& 	64\\
NGC0055& 	9& 	2.11& 	0.11&   	2& 	77\\
NGC0247& 	7& 	3.7& 	0.19&   	2& 	74\\
NGC0300& 	7& 	2.08& 	0.1&    	2& 	42\\
NGC0891& 	3& 	9.91& 	0.5&    	2& 	90\\
NGC1705& 	11& 	5.73& 	0.29& 	2& 	80\\
NGC2366& 	10& 	3.27& 	0.16& 	2& 	68\\
NGC2403& 	6& 	3.16& 	0.16&   	2& 	63\\
NGC2683& 	3& 	9.81& 	0.49&   	2& 	80\\
NGC2841& 	3& 	14.1& 	1.4&    	3& 	76\\
NGC2915& 	11& 	4.06& 	0.2& 	2& 	56\\
NGC2976**& 	5& 	3.58& 	0.18&   	2& 	61\\
NGC3109& 	9& 	1.33& 	0.07&   	2& 	70\\
NGC3198& 	5& 	13.8& 	1.4&    	3& 	73\\
NGC3741& 	10& 	3.21& 	0.17& 	2& 	70\\
NGC4068& 	10& 	4.37& 	0.22& 	2& 	44\\
NGC4214& 	10& 	2.87& 	0.14& 	2& 	15\\
NGC5055& 	4& 	9.9& 	0.3&    	2& 	55\\
NGC5907& 	5& 	17.3& 	0.9&    	2& 	88\\
NGC6503& 	6& 	6.26& 	0.31&   	2& 	74\\
NGC6789& 	11& 	3.52& 	0.18& 	2& 	43\\
NGC6946& 	6& 	5.52& 	1.66&   	2& 	38\\
NGC7331& 	3& 	14.7& 	1.5&    	3& 	75\\
NGC7793& 	7& 	3.61& 	0.18&   	2& 	47\\
NGC7814& 	2&  	14.4& 	0.66& 	2& 	90\\
PGC51017**& 	11& 	13.6& 	1.4& 	2& 	66\\
UGC01281& 	8&   	5.27& 	0.24& 	2& 	90\\
UGC04305**& 	10& 	3.45& 	0.17& 	2& 	40\\
UGC04483& 	10& 	3.34& 	0.31& 	2& 	58\\
UGC07151& 	6& 	6.87& 	0.34&   	2& 	90\\
UGC07232& 	10& 	2.83& 	0.17& 	2& 	59\\
UGC07524& 	9& 	4.74& 	0.24& 	    2& 	46\\
UGC07559& 	10& 	4.97& 	0.25& 	2& 	61\\
UGC07577& 	10& 	2.59& 	0.13& 	2& 	63\\
UGC07866& 	10& 	4.57& 	0.23& 	2& 	44\\
UGC08286& 	6& 	6.5& 	0.21&   	2& 	90\\
UGC08490& 	9& 	4.65& 	0.53&   	2& 	50\\
UGC08837& 	10& 	7.21& 	0.36& 	2& 	80\\
UGCA281& 	11& 	5.68& 	0.28& 	2& 	67\\
UGCA442& 	9& 	4.35& 	0.22& 	    2& 	64\\
UGCA444& 	10& 	0.98& 	0.05& 	2& 	78\\
    \hline \hline           
      \end{tabular}
      \caption{Table from  \citet{2016Lelli}. 
      (1) Hubble type:  from (CITE) de
Vaucouleurs et al. (1991), (CITE) Schombert et al. (1992), or (CITE) NED3
of: 0 = S0, 1 = Sa, 2 = Sab,
3 = Sb, 4 = Sbc, 5 = Sc, 6 = Scd, 7 = Sd, 8 = Sdm,
9 = Sm, 10 = Im, 11 = BCD.  (2) Distance method: : 1 = Hubble flow,
2 = tip of the red giant branch, 3 = Cepheids, 4 = Ursa Major
cluster of galaxies, 5 = supernovae. ** notes galaxies excluded from fit bc fail to fit, and see if we want to include SN gals, there's only 3...and 2 of the 3 that fail are weirdi rotations (non-circular)}
0      \label{tab:Tset}
  \end{table*}
  
  %%NOTE: Homies for SPARC used hybrid rotation curves on 56 galaxies, combining H-alpha  (inner) with HI (outer) Check if our "bad" gals are part of thi, or if is algo mas. Despite the high spatial resolution, Hα rotation curves are sometimes quite uncertain due to the patchy distribution and complex kinematics of Hα gas, especially for low-mass and LSB galaxies. We %use only H I data for the several objects:DDO 154, IC 2574, NGC 3109, NGC 5033, NGC 5055,NGC 6946, NGC 7331, and UGC 2259.(THESE ARE ALL GOOD
  %H I and Hα rotation
%curves, however, agree within the errors

%or face-on ones.We assign a quality flag (Q) to each rotation curve using the following scheme: Q = 1 for galaxies with high-quality H I data or hybrid Hα/H I rotation curves (99 objects); Q = 2 for
%galaxies with minor asymmetries and/or H I data of lower
%quality (64 objects); Q = 3 for galaxies with major asymmetries, strong noncircular motions, and/or offsets between H I
%nd stellar distributions (12 objects). Galaxies with Q = 3 are%
%not suited for detailed dynamical studies: we build mass models
%for completeness but do not consider them in our analysis.
%K BIEN_ Both UGC04305_rotmod and PGC51017_rotmod have Q=3. discard anyhow. but not: NGC2976_rotmod, it's a Q=2. Plus we have a few other galaxies in this subset with a Q=3,,,


  We search the free parameter $\alpha$ space of this subset of SPARC galaxies by plotting the single valued free parameter of each galaxy   versus a proxy for the gradient in the gravitational potential
  
   \begin{equation}
     \rho = L_{total}/R_{eff}.
 \end{equation}
 
 The  estimates of the  total luminosities $L_{total}$ at $3.6 \mu m$,  assume a solar
absolute magnitude of 3.24 at $3.6 \mu m$ (CITE Oh et al. 2008). The effective radius $R_{eff}$ is    the radius encompassing half of the total luminosity \citet{2016Lelli}.  
 

 We find $\alpha$  to be   strongly correlated to  this quantity. 
  
 \begin{figure*}[h]
\scalebox{0.5}%
{\includegraphics{alpha_10_18_22.png} }
\caption{ Remake this graph, nice   }
\label{alpha2}
\end{figure*}  
 
for 
\begin{equation}
    \alpha = 
\end{equation}



We might assume that $\alpha$ is a ratio of the same quantity for  the Milky Way $\rho_{mw}$ with respect to the other   galaxy $\rho_{gal}$  

\begin{equation}
\alpha=\left(\frac{\rho_{mw}}{\rho_{gal}}\right)^{b}  ,
\label{correl}
\end{equation}

then we arrive at a similar fit (See Fig.~\ref{alpha1}.

 
\subsection{MOND  \& RAR Comparison}



 
 In MOND,  classical gravity is transitioned to  a paradigm where the luminous mass is the only mass  but    the acceleration scale of gravity changes on the enormous distance scales of galaxies. MOND amounts to a phenomenological law of galaxy rotation curves, though does struggle in some galaxies.  Relativistic extensions of MOND have been summarized by  \citet{Famaey2012} and include notably the work of CITE(BEKENSTEIN), though all approaches require modifications to Einstein's well tested gravitational theory (CITE). 
  BY transitioning the MONDian concept of changing acceleration scales to changing  relative curvatures, we implicitly replace MOND's acceleration scale with the baryonic potential of the Milky Way, and find a more 
  efficient and flexible model than MOND. We are successful at fitting galaxies at Cepheid distances (GIVE NAME) that MOND struggles with, and we propose that this increased power is due to the more flexible use of the changing acceleration scale when pinned to the Milky Way and allowed to be the other frame in all transformations. 
  
While reported error    estimates on rotation curve velocities  have not been standardized across the field~\citep{Blok,Gent},     one can reliably   compare fits to the same data with the same  errors. The     reduced $\chi^2_r$ values printed on each graph and in table (MAKE TABLE), demonstrate the higher efficacy of the RCFM. 
 
  \citet{McGaugh2016RAR} is the paper where they present new fitting formula (compare to MOND), but en verdad, no me importa a mi. 
  
 
 
 
 
% \cite{McGaugh2016RAR} is the other one. This one describes the sample. 175 SPARC galaxies. Describes all the assumptions for the disk, gas, bulge models, the band used and reasoning, rotation curve data from HI, 
 
% They only fit a subset (153 of 175) of these galaxies.I get 128, I guess I'm cutting endpoints of 85 and 30 and they include. Plus I exclude like five that I think fail to fit
 
 
 \section{  Conclusions \label{sec:conclu}  }
 

{\color{red} This heuristic replacement of dark matter is not a fundamental derivation, but    one supposes if it is possible to quantify these
excesses in shifted spectra using only luminous mass as we do in this paper,  then a       derivation from first principles should   exist.  The curious conspiracy of the luminous mass to   determine the dark matter  is here more compactly explained and characterized as the imprint of our Milky Way on Doppler shifted spectra we receive from external galaxies. Not only is this a more efficient way to predict rotations curves as judged by the average reduced $\chi^2$, but it is also a  more conservative choice than either MOND or dark matter which both require new physics \cite{de_Blok_2010}.  
  
 
 

  
 
%Places where dark matter is invoked. Structure formation: put spherical model replace flat, solve. Galaxy clusters, same,put flow lines on surfaces supported by dark energy. Tully Fisher: temp dependence with mass - but read Beckenstein's description carefully. Hemakes some points.  
   
 
We   present here a new picture of flat-rotation curves which does not   modify Einstein gravitational physics, but adds a new technique to the standard transformations of   relativity. }
  
 

%%%%%%
%%%%%
%%%
%%
%
%%%%%%


 % \caption{Results   for SPARC Luminous mass profiles  [NOT UPDATED]\label{sumRESULTS}}
 % \begin{tabular}{@{}llccccccc@{}}
 % \hline
 %  Galaxy     	  &Ref.~&  \multicolumn{3}{c}{\underline{ Other Model Fit Results}}	& & \multicolumn{3}{c}{\underline{LCM Fit Results }}  \\
%\hline
%   	 	& & &  $M/L_{disk}$		& $\chi^2_r$	&&$M/L$&$r_e$&$\chi^2_r$ \\ 
% \hline
%F 563-1	& 2	%& 			
%					&NFW	&--	 		&-- & 	&1.13 	&2.84 	&0.06 \\
%M 31* 	& 12 	%&7.5
%					&ISO	 	 &7.50		&0.36 & 	&5.88	 &4.80 	&0.04  \\
%M 33		& 5	 %&	K 		
%					& NFW 	&0.70		&2.46&  	 &1.98	 &1.46	& 0.17 \\
%NGC 891*	& 11 	% &3.6$\,\umu$m 
		%			&MaxLight &0.9 			&1.10  & 	 &--  	 &	&0.25 \\ 
%NGC 925 	&3	%&3.6$\,\umu$m
 			%		&ISO 	&0.18		 &2.40& 	&0.92	 &4.35	&0.11 \\
%NGC 2403	 & 3	%& 3.6$\,\umu$m
		%			& NFW	&0.41 	 	&4.56& 	&1.12	 	&2.18		& 0.88 \\
%NGC 2841*  &6-3	% & 3.6$\,\umu$m
%					&  James	&0.74 	 	&0.45 & 	&6.25	  &4.84	&0.11 \\
%NGC 2903  &10	 %&B	   		
%					&MOND   	&3.60 	 	&10.71& 	 &2.2	 	&2.81		&0.47\\
% NGC 3198 & 3 	%&3.6$\,\umu$m
%					 &NFW	&0.80	   	&5.40  &   	 &1.80 	 &5.10	&0.64   \\
%NGC 3521  & 8-6	 %&3.6$\,\umu$m 
%					&MOND	&0.71 	 	&0.97 & 	 &2.13  &3.23	&0.22 \\
%NGC 3726	& 10	%&B 			
%					&MOND	&1.00	 	&3.57& 	&1.06	 &2.70	&0.05 \\
%NGC 3953	& 10	%&B 			
%					&MOND	&2.7		 	&1.35& 	&1.79 	&2.60 	&0.35 \\
%NGC 3992	& 10	%&B 			
%					&MOND	&4.93	 	&0.50& 	&2.45 	 &4.77	&0.04 \\
%NGC 4088	& 10	%&B 			
%					&MOND	&1.16		 	&1.70& 	&5.58	 &2.70	&0.27 \\
%NGC 4138	& 10	%&B 			
%					&MOND	&3.5		 	&2.12& 	&3.67	 &1.46	&0.01 \\
%NGC 5055*	 & 3	%&3.6$\,\umu$m 
%					&  NFW	&0.79	 	&17.23&  	&5.87	 &3.29	&0.69 \\
%NGC 5533*	 & 10 %&B 			
%					&MOND	&0.6		  	&1.57 & 	&7.11 	  &3.23	&0.22  \\
%NGC 5907*	& 10	%&B 			
				%	&MOND	&1.6 		 	 &0.44& 	&2.04	 &3.45	%&0.09 \\
%NGC 6946*	&  10	 %& B			
%					&  MOND	& 0.5		 	&   3.03& 	&1.44 	 &0.76	&0.07  \\ 
%NGC 6946*	&  3	 %& B			
%					&  NFW	& 0.5		 	&   3.03& 	&1.44 	 &0.76	&0.07  \\ 
%NGC 7331	&8	% & 3.6$\,\umu$m
%					&James	&0.4 		 	& 0.45& 	&1.34	 &2.44	&0.09 \\
%NGC 7793  &14	%&B			
%					&ISO		&2.6		 	&1.08& 	&2.7	 &1.51	&0.11 \\
%NGC 7814* &11 	% & 3.6$\,\umu$m
%					&ISO   	& 0.68  	 	& 0.25& 	 &-- 	   &	&0.20 \\ 
%UGC 128		&6	%&				
	%				&James	&			&	&	&1.58	  &10.3	%&0.20\\
%UGC 6973	& 10	%&B 			
%					&MOND	&2.7		 	&23.5 & 	&--  	& 	&0.06 \\
%UGC 7524	& 6	%&B 			
%					&James	&--		 	&-- & 	&2.10 	&3.32 	&0.06 \\
% 1.~\citet{Bege}, 2.~\citet{JNav}, 3.~\citet{Blok} , 4.~\citet{Maria}, 
%5.~\citet{Cor03}, 6.~\citet{James},   7.~\citet{Batt},   8.~\citet{Gent},   9.~\citet{Bot},   10.~\citet{SanMcGa},
 % 11.~\citet{Frat},   12.~\citet{Car},   13.~\citet{giraud2000universal},   14.~\citet{Dicaire}, %15.~\cite{Klypin}. \\
%    \end{minipage}
%\end{table*} 


   

  \section[]{Acknowledgments}
 This work is dedicated to Emmett Till. This paper was written with gratitude on the usual, and accustomed  territories
of the Coast Salish, Cheyenne, Arapaho and Ute Tribal Peoples.
  The authors would like to thank  V.\,P.\,  Nair,   R.\, Walterbos, S.\ McGaugh, A.\, Klypin, K. Bender, C. Beetle and     T.\, Boyer.   \\
  
 
%The   dark matter problem  is currently woven into most faces of our cosmology (weaklensing, galaxy rotation curves, early structure formation, galaxy and cluster interactions).   We posit that not all of these problems are actually the same physics problem.  We will address in this paper only the dark matter problem in spiral galaxies, the so-called flat-rotation curve problem discovered by \citet{1978Rubin,Bosma78}. We believe the ideas presented here are extensible to the problem of galaxy and cluster interactions, and perhaps weak lensing, but not to early structure formation, which we believe to be a different physics problem. 


 \begin{figure}[h]
\begin{subfigure}{.5\textwidth}
  \centering
  \includegraphics[width=.8\linewidth]{NGC5055_deBlok_XueSofue}
  \caption{deBlok\cite{Blok1}}
  \label{fig:sfig1}
\end{subfigure}%
\begin{subfigure}{.5\textwidth}
  \centering
  \includegraphics[width=.8\linewidth]{NGC5055_rotmod-Copy1_XueSofue}
  \caption{SPARC\cite{2016Lelli}}
  \label{fig:sfig2}
\end{subfigure}
\begin{subfigure}{.5\textwidth}
  \centering
  \includegraphics[width=.8\linewidth]{NGC5055_deBlok-lcmEdit5_XueSofue}
  \caption{Lum edits5 \cite{Blok1}}
  \label{fig:sfig3}
\end{subfigure}
\caption{RCFM fits of NGC 5055 }
\label{fig:fig5055}
\end{figure}
 
 
 
  \begin{figure}[h]
\begin{subfigure}{.5\textwidth}
  \centering
  \includegraphics[width=.8\linewidth]{NGC6946_deBlok_TH_XueSofue}
  \caption{deBlok\cite{Blok1}}
  \label{fig:sfig4}
\end{subfigure}%
\begin{subfigure}{.5\textwidth}
  \centering
  \includegraphics[width=.8\linewidth]{NGC6946_rotmod-Copy1_XueSofue}
  \caption{SPARC\cite{2016Lelli}}
  \label{fig:sfig5}
\end{subfigure}
\caption{RCFM fits  of NGC 6946 }
\label{fig:fig6946}
\end{figure}
%
%
%

  \begin{figure}[h]
\begin{subfigure}{.5\textwidth}
  \centering
  \includegraphics[width=.8\linewidth]{NGC2841_deBlokTHINGS_XueSofue}
  \caption{deBlok\cite{Blok1}}
  \label{fig:sfig9}
\end{subfigure}%
\begin{subfigure}{.5\textwidth}
  \centering
  \includegraphics[width=.8\linewidth]{NGC2841_rotmod-Copy1_XueSofue}
  \caption{SPARC\cite{2016Lelli}}
  \label{fig:sfig10}
\end{subfigure}
\caption{RCFM fits  of NGC 2841}
\label{fig:fig2841}
\end{figure}

\clearpage
  \begin{figure}[h]
\begin{subfigure}{.5\textwidth}
  \centering
  \includegraphics[width=.8\linewidth]{NGC3198_deBlok_XueSofue}
  \caption{deBlok \cite{Blok}}
  \label{fig:sfig6}
\end{subfigure}%
\begin{subfigure}{.5\textwidth}
  \centering
  \includegraphics[width=.8\linewidth]{NGC3198_GentileMaria_XueSofue}
  \caption{Gentile \cite{Maria}}
  \label{fig:sfig7}
\end{subfigure}
\begin{subfigure}{.5\textwidth}
  \centering
  \includegraphics[width=.8\linewidth]{NGC3198_rotmod-Copy1_XueSofue}
  \caption{SPARC\cite{2016Lelli}}
  \label{fig:sfig8}
\end{subfigure}
\caption{RCFM fits  of NGC 3198}
\label{fig:fig3198}
\end{figure}
 %   
%
% 
\clearpage
%%%

  \begin{figure}[h]
\begin{subfigure}{.5\textwidth}
  \centering
  \includegraphics[width=.8\linewidth]{NGC7793_DicaireISO_XueSofue}
  \caption{Dicaire \cite{Dicaire1}}
  \label{fig:sfig11}
\end{subfigure}%
\begin{subfigure}{.5\textwidth}
  \centering
  \includegraphics[width=.8\linewidth]{NGC7793_GentileTaM_XueSofue}
  \caption{Gentile\cite{Gent}}
  \label{fig:sfig12}
\end{subfigure}
\begin{subfigure}{.5\textwidth}
  \centering
  \includegraphics[width=.8\linewidth]{NGC7793_rotmod-Copy1_XueSofue}
  \caption{SPARC\cite{2016Lelli}}
  \label{fig:sfig13}
\end{subfigure}
\caption{RCFM fits  of NGC 7793}
\label{fig:fig7793}
\end{figure}
%
%
%
%
\clearpage
  \begin{figure}[h]
\begin{subfigure}{.5\textwidth}
  \centering
  \includegraphics[width=.8\linewidth]{NGC2903deBlok_XueSofue.png}
  \caption{deBlok\cite{Blok1}}
  \label{fig:sfig14}
\end{subfigure}%
\begin{subfigure}{.5\textwidth}
  \centering
  \includegraphics[width=.8\linewidth]{NGC2903_GentileTaM_XueSofue.png}
  \caption{Gentile\cite{Gent}}
  \label{fig:sfig15}
\end{subfigure}
\begin{subfigure}{.5\textwidth}
  \centering
  \includegraphics[width=.8\linewidth]{NGC2903_rotmod-Copy1_XueSofue.png}
  \caption{SPARC\cite{2016Lelli}}
  \label{fig:sfig16}
\end{subfigure}
\caption{plots of NGC 2903}
\label{fig:fig2903}
\end{figure}
%
\clearpage
%
%  
  \begin{figure}[h]
\begin{subfigure}{.5\textwidth}
  \centering
  \includegraphics[width=.8\linewidth]{NGC2915_rotmod-Copy1_XueSofue.png}
  \caption{SPARC\cite{2016Lelli}}
  \label{fig:sfig17}
\end{subfigure}%
\begin{subfigure}{.5\textwidth}
  \centering
  \includegraphics[width=.8\linewidth]{NGC2915_San96_XueSofue.png}
  \caption{Sanders\cite{San96}}
  \label{fig:sfig18}
\end{subfigure}
\caption{plots of NGC 2915}
\label{fig:fig2915}
\end{figure}
%
%
%
%  
%%%%%%%
  \begin{figure}[h]
\begin{subfigure}{.5\textwidth}
  \centering
  \includegraphics[width=.8\linewidth]{NGC3521_GentileTaM_XueSofue.png}
  \caption{Gentile\cite{Gent}}
  \label{fig:sfig19}
\end{subfigure}%
\begin{subfigure}{.5\textwidth}
  \centering
  \includegraphics[width=.8\linewidth]{NGC3521_rotmod-Copy1_XueSofue.png}
  \caption{SPARC\cite{2016Lelli}}
  \label{fig:sfig20}
\end{subfigure}
\caption{plots of NGC 3521}
\label{fig:fig3521}
\end{figure}
%
%
%
\clearpage
 \begin{figure}[h]
\begin{subfigure}{.5\textwidth}
  \centering
  \includegraphics[width=.8\linewidth]{NGC7331_GentileTaM_XueSofue.png}
  \caption{Gentile\cite{Gent}}
  \label{fig:sfig21}
\end{subfigure}%
\begin{subfigure}{.5\textwidth}
  \centering
  \includegraphics[width=.8\linewidth]{NGC7331_rotmod-Copy1_XueSofue.png}
  \caption{SPARC\cite{2016Lelli}}
  \label{fig:sfig22}
\end{subfigure}
\caption{plots of NGC 7331}
\label{fig:fig7331}
\end{figure}
%%%%%%%
 \begin{figure}[h]
\begin{subfigure}{.5\textwidth}
  \centering
  \includegraphics[width=.8\linewidth]{F563_1deBlok_XueSofue.png}
  \caption{de Blok}
  \label{fig:sfig23}
\end{subfigure}%
\begin{subfigure}{.5\textwidth}
  \centering
  \includegraphics[width=.8\linewidth]{F563_1McGaugh_XueSofue.png}
  \caption{McGaugh}
  \label{fig:sfig24}
\end{subfigure}
\caption{plots of F563-1}
\label{fig:figF563-1}
\end{figure}
\clearpage
%%%%%%
%%%%%%%%
%%%%%%
%%%%%%%
%%%%%%
  
 
\begin{figure*}
    \centering
    \includegraphics{MW_Enbang_Li}
    \caption{Enbang Li \cite{Li2016ModellingMD}}
    \label{fig:my_label}
\end{figure*}

    
     

\bibliography{LCM} 

\end{document}
%
% ****** End of file apssamp.tex ******